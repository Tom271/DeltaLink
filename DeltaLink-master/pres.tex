\documentclass[11pt]{beamer}
\usepackage[utf8]{inputenc}
\usepackage[T1]{fontenc}
\usepackage{lmodern}
\usepackage{amsmath}
\usepackage{amsfonts}
\usepackage{amssymb}
\usepackage{graphicx}
\usepackage{subcaption}
\usetheme{Berlin}
\usepackage{palatino}
\graphicspath{{/home/tom/Documents/RandomGraphs/uptodate0205/generating-heterogeneous-networks/Figures/}}
\AtBeginSection[]
{
    \begin{frame}
    \frametitle{Table of Contents}
    \tableofcontents[currentsection]
\end{frame}
}
\begin{document}
    \author{Thomas M. Hodgson}
    \title{Random Graph Simulation}
    %\subtitle{}
    %\logo{}
    \institute{School of Mathematics, University of Manchester}
    %\date{}
    %\subject{}
    %\setbeamercovered{transparent}
    %\setbeamertemplate{navigation symbols}{}
    \begin{frame}[plain]
    \maketitle
\end{frame}

\begin{frame}
\frametitle{Contents}
\tableofcontents
\end{frame}
\section[ERGM]{Exponential Random Graph Model}
\begin{frame}
\frametitle{Observed Networks}
\begin{figure}
\centering
%\includegraphics[height=0.65\linewidth]{socnet.png}
\end{figure}
\end{frame}

\begin{frame}
\frametitle{Random Graph Models}
\begin{align*}\mathbb{P}(Y=y) = \frac{\exp(H(y))}{Z}\end{align*}
\end{frame}


\section[Simulation Methods]{Simulation Methods}
\begin{frame}
\frametitle{One Link Algorithm}
\begin{figure}
    \centering
    %\includegraphics[height=0.6\linewidth]{onelinkconv.png}
\end{figure}
\end{frame}

\begin{frame}
\frametitle{Delta Link Algorithm with \(W=2\)}
\begin{figure}
    \centering
    %\includegraphics[height=0.6\linewidth]{window1.png}
\end{figure}
\end{frame}

\begin{frame}

\frametitle{Delta Link Algorithm with \(W=2\)}
\begin{figure}
    \centering
    %\includegraphics[height=0.6\linewidth]{histfitwindow1.png}
\end{figure}
\end{frame}

\begin{frame}
\frametitle{Delta Link Algorithm with Varying Window Width}
\begin{figure}
    \begin{subfigure}{.4\textwidth}
        \centering
        %\includegraphics[width=\linewidth]{Changed_Hitting_Time/window2.png}
        \label{window2nohit}
    \end{subfigure} %
    \begin{subfigure}{.4\textwidth}
        \centering
        %\includegraphics[width=\linewidth]{Changed_Hitting_Time/window3.png}
        \label{window3nohit}
    \end{subfigure}\\%
    \begin{subfigure}{.4\textwidth}
        \centering
        %\includegraphics[width=\linewidth]{Changed_Hitting_Time/window10.png}
        \label{window10nohit}
    \end{subfigure} %
    \begin{subfigure}{.4\textwidth}
        \centering
        %\includegraphics[width=\linewidth]{Changed_Hitting_Time/window20.png}

        \label{window20nohit}
    \end{subfigure} 

\end{figure}
\end{frame}

\begin{frame}

\frametitle{Time to Convergence and Computation Time}
\begin{figure}
    \centering
    %\includegraphics[height=0.6\linewidth]{comptime.png}
\end{figure}
\end{frame}

\begin{frame}
\frametitle{Image References}
Social Network -- \url{http://vis.cs.ucdavis.edu/papers/social_networks.pdf}\\
Scale-Free Network -- \url{http://estebanmoro.org/2015/12/temporal-networks-with-r-and-igraph-updated/}\\
Watts-Strogatz Network -- \url{https://commons.wikimedia.org/wiki/File:Watts_strogatz.svg}\\
\end{frame}
\end{document}